\section{Reaction}
A photochemical reaction typically is:
\begin{equation}
\renewcommand{\arraystretch}{1.2}
\ce{CH4 + \h{\nu} ->[\sigma(\lambda)]}\left\{\begin{array}{l}
                    \ce{-> ^1CH2 + H2} \\
                    \ce{-> CH3 + H}\\
                    \ce{-> CH2 + 2 H}\\
                    \ce{-> CH4+}\\
                    \ce{-> CH3+ + H}\\
                    \ce{-> CH2+ + H2}\\
                    \ce{-> CH+ + H2 + H}\\
                    \ce{-> H+ + CH3}\\
                    \ce{-> CH + H2 + H}\\
                    \end{array}\right.
\end{equation}
The rate constant is computed through the equation
\begin{equation}
k(z) = \int_0^\infty \cs(\lambda)\fhv^{\h\nu}(z,\lambda)\dd\lambda
\end{equation}
with $\fhv^{\h\nu}(z,\lambda)$ the photon flux at the considered altitude.
Thus the calculations is made in three steps:
\begin{enumerate}
\item calculate the function $\fhv^{\h\nu}(z,\lambda)$,
\item harmonize the $\lambda$ grid,
\item integrate.
\end{enumerate}

\section{Photon flux at altitude \texorpdfstring{$z$}{z}}

\begin{equation}
\fhv^{\h\nu}(z,\lambda) = \fhvtop^{\h\nu}(\lambda) \exp\left(-\tau(z,\lambda)\right)
\end{equation}
with \fhvtop\ the photon flux at the top of the atmosphere, and
\begin{equation}
\tau(z,\lambda) = \chapman\sum_s\sigma_s(\lambda)\int_z^\infty n_s(z')\dd z'
\end{equation}
with $n_s(z')$ the molar density of species $s$ at altitude $z'$.
the Chapman function depends on the
incident angle $\chi$.\\
For $\chi \le 75^\circ$
\begin{equation}
\chapman =  \frac{1}{\cos(\chi)},
\end{equation}
for $75^\circ < \chi \le 90^\circ$
\begin{equation}
\chapman = \sqrt{\frac{\pi \x}{2}} 
                \left(1 - \erf\left(\sqrt{\frac{\x}{2}}|\cos(\chi)|\right)\right)\exp\left(\frac{\x}{2}\cos^2(\chi)\right),
\end{equation}
for $90^\circ < \chi$
\begin{equation}
\begin{split}
\chapman = & \sqrt{2\pi \x}\bigg[\sqrt{\sin(\chi)}\exp\left(\x\left(1-\sin(\chi)\right)\right) \\
           &  -\frac{1}{2}\exp\left(\frac{\x}{2}\cos^2(\chi)\left(1-\erf\left(\sqrt{\frac{\x}{2}}|\cos(\chi)|\right)\right)\right)\bigg]
\end{split}
\end{equation}
And the definition of \x\ is given by
\begin{equation}
\x(z) = \frac{\RTitan + z}{\sch(z)}
\end{equation}
with $\sch(z)$ being the scale height of definition
\begin{equation}
\sch(z) = \frac{\kb T(z)}{\gloc(z)\mean{\mass}}
\end{equation}
with \kb\ the Boltzmann constant, 
\mean{\mass}\ the atmospheric mass ($\mean{\mass} = \frac{\Mm(z)}{\Nav}\conc_\text{tot}(z)$, 
$\Mm(z)$ the molar mass of the atmosphere at altitude $z$,
\Nav\ the Avogadro number, $\conc_\text{tot}(z)$ the density at altitude $z$), 
$\gloc(z)$ the gravitationnal field at altitude $z$, given by the equation
\begin{equation}
\gloc(z) = \frac{\Guni\cdot\MTitan}{\left(\RTitan + z\right)^2}
\label{gfield}
\end{equation}

\section{Photon flux at the top of the atmosphere}
\begin{equation}
\fhvtop(\lambda) = \frac{\fhvAU}{\dSS^2}
\end{equation}
with \dSS\ the distance between the Sun and Saturn.

\section{Finally}

The whole computations starts with the inputs:
\Guni, the universal gravitationnal constant,
\MTitan, the mass of Titan,
\RTitan, the radius of Titan,
\Nav, the Avogadro number,
\kb, the Boltzmann constant,
$\{n_s\}$, the molar densities of the species at each altitude,
\fhvAU, the solar flux at 1 AU of the Sun,
\dSS, the distance Sun-Saturn,
$\chi$, the zenith angle,
$\{\cs_s(\lambda)\}$, the cross-section of the species.
\begin{center}
\begin{tabular}{rrrrrrrr}
\Guni   & \rdelim\}{3}{28pt}[\gloc(z)] & & & & & & \\
\MTitan &      & \rdelim\}{7}{35pt}[\sch(z)] & & & & &  \\
\RTitan &      & &                       &    & & & \\
        & \Nav & \RTitan & \rdelim\}{2}{27pt}[\x(z)]&\rdelim\}{3}{58pt}[\chapman] & & & \\[5pt]
        & \kb  & & $\chi$                &             & \rdelim\}{4}{39pt}[$\tau(z,\lambda)$]&\\
        & $T(z)$ & &                     & \sch(z)     & & \rdelim\}{5}{41pt}[$\fhv(z,\lambda)$]\\
\multirow{2}{*}{$\{n_s\}$} 
        & \rdelim\}{2}{28pt}[\Mm] & &   & $\{n_s\}$   & &\\
        &      & &                       & $\{\cs_s\}$ & &\\[5pt]
        &      & &                       & $\fhvAU(\lambda)$ & \rdelim\}{2}{38pt}[$\fhvtop(\lambda)$] &\\
        &      & &                       & \dSS        & &
\end{tabular}
\end{center}


\begin{figure}
\centering
\includegraphics[width=\textwidth]{sunray}
\caption{\label{sunray}Titan under the sun (from \citet[Fig.~3.3]{Haye2005})}
\end{figure}

\section{The data}

Starting with $\fhvAU(\lambda)$, thus values of
irradiance on a grid of wavelengthes. We calculate
$\fhv(z,\lambda)$ on the same grid with the given
workflow. Then we need to integrate $\cs(\lambda)\fhv(z,\lambda)$
for every altitude.

\begin{figure}
\centering
\includegraphics{cs_CH4}
\caption{\label{csCH4}Cross-section of \ce{CH4}}
\end{figure}

